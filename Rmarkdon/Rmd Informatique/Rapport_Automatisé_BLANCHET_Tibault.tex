% Options for packages loaded elsewhere
\PassOptionsToPackage{unicode}{hyperref}
\PassOptionsToPackage{hyphens}{url}
\documentclass[
]{article}
\usepackage{xcolor}
\usepackage[margin=1in]{geometry}
\usepackage{amsmath,amssymb}
\setcounter{secnumdepth}{-\maxdimen} % remove section numbering
\usepackage{iftex}
\ifPDFTeX
  \usepackage[T1]{fontenc}
  \usepackage[utf8]{inputenc}
  \usepackage{textcomp} % provide euro and other symbols
\else % if luatex or xetex
  \usepackage{unicode-math} % this also loads fontspec
  \defaultfontfeatures{Scale=MatchLowercase}
  \defaultfontfeatures[\rmfamily]{Ligatures=TeX,Scale=1}
\fi
\usepackage{lmodern}
\ifPDFTeX\else
  % xetex/luatex font selection
\fi
% Use upquote if available, for straight quotes in verbatim environments
\IfFileExists{upquote.sty}{\usepackage{upquote}}{}
\IfFileExists{microtype.sty}{% use microtype if available
  \usepackage[]{microtype}
  \UseMicrotypeSet[protrusion]{basicmath} % disable protrusion for tt fonts
}{}
\makeatletter
\@ifundefined{KOMAClassName}{% if non-KOMA class
  \IfFileExists{parskip.sty}{%
    \usepackage{parskip}
  }{% else
    \setlength{\parindent}{0pt}
    \setlength{\parskip}{6pt plus 2pt minus 1pt}}
}{% if KOMA class
  \KOMAoptions{parskip=half}}
\makeatother
\usepackage{graphicx}
\makeatletter
\newsavebox\pandoc@box
\newcommand*\pandocbounded[1]{% scales image to fit in text height/width
  \sbox\pandoc@box{#1}%
  \Gscale@div\@tempa{\textheight}{\dimexpr\ht\pandoc@box+\dp\pandoc@box\relax}%
  \Gscale@div\@tempb{\linewidth}{\wd\pandoc@box}%
  \ifdim\@tempb\p@<\@tempa\p@\let\@tempa\@tempb\fi% select the smaller of both
  \ifdim\@tempa\p@<\p@\scalebox{\@tempa}{\usebox\pandoc@box}%
  \else\usebox{\pandoc@box}%
  \fi%
}
% Set default figure placement to htbp
\def\fps@figure{htbp}
\makeatother
% definitions for citeproc citations
\NewDocumentCommand\citeproctext{}{}
\NewDocumentCommand\citeproc{mm}{%
  \begingroup\def\citeproctext{#2}\cite{#1}\endgroup}
\makeatletter
 % allow citations to break across lines
 \let\@cite@ofmt\@firstofone
 % avoid brackets around text for \cite:
 \def\@biblabel#1{}
 \def\@cite#1#2{{#1\if@tempswa , #2\fi}}
\makeatother
\newlength{\cslhangindent}
\setlength{\cslhangindent}{1.5em}
\newlength{\csllabelwidth}
\setlength{\csllabelwidth}{3em}
\newenvironment{CSLReferences}[2] % #1 hanging-indent, #2 entry-spacing
 {\begin{list}{}{%
  \setlength{\itemindent}{0pt}
  \setlength{\leftmargin}{0pt}
  \setlength{\parsep}{0pt}
  % turn on hanging indent if param 1 is 1
  \ifodd #1
   \setlength{\leftmargin}{\cslhangindent}
   \setlength{\itemindent}{-1\cslhangindent}
  \fi
  % set entry spacing
  \setlength{\itemsep}{#2\baselineskip}}}
 {\end{list}}
\usepackage{calc}
\newcommand{\CSLBlock}[1]{\hfill\break\parbox[t]{\linewidth}{\strut\ignorespaces#1\strut}}
\newcommand{\CSLLeftMargin}[1]{\parbox[t]{\csllabelwidth}{\strut#1\strut}}
\newcommand{\CSLRightInline}[1]{\parbox[t]{\linewidth - \csllabelwidth}{\strut#1\strut}}
\newcommand{\CSLIndent}[1]{\hspace{\cslhangindent}#1}
\setlength{\emergencystretch}{3em} % prevent overfull lines
\providecommand{\tightlist}{%
  \setlength{\itemsep}{0pt}\setlength{\parskip}{0pt}}
\usepackage{bookmark}
\IfFileExists{xurl.sty}{\usepackage{xurl}}{} % add URL line breaks if available
\urlstyle{same}
\hypersetup{
  pdftitle={Rapport Informatique Automatisé},
  pdfauthor={BLANCHET Tibault},
  hidelinks,
  pdfcreator={LaTeX via pandoc}}

\title{Rapport Informatique Automatisé}
\author{BLANCHET Tibault}
\date{06 février 2026}

\begin{document}
\maketitle

{
\setcounter{tocdepth}{3}
\tableofcontents
}
\section{Problèmatique}\label{probluxe8matique}

Quesque le Cloud et les NAS? Ont-t-ils un intéret pour le grand-public ?
Quelles solutions sont disponibles sur le marché ?

\section{Définitions}\label{duxe9finitions}

Un peu de définition le Cloud est ``Le terme « cloud » désigne les
serveurs accessibles sur Internet, ainsi que les logiciels et bases de
données qui fonctionnent sur ces serveurs.'' {[}Auteur ?{]}
\emph{Qu'est-ce que le cloud ? {\textbar} {Définition} du cloud}
({[}sans date{]}) Un NAS est ``Un système de NAS est un dispositif de
stockage de grande capacité connecté à un réseau qui permet aux
utilisateurs et aux clients autorisés du réseau de stocker et de
récupérer des données à partir d'un emplacement centralisé.'' (HPE,
{[}sans date{]})

\subsection{Présentation Générale}\label{pruxe9sentation-guxe9nuxe9rale}

La notion de Cloud et de NAS sont fortement interconnectés pour les
utilisations personnelles, en effet il s'agit d'une solution de stockage
d'information à forte capacité disponible à distance par l'utilisateur.
Un appareil NAS fonctionne sur tout type de plateforme ou de système
d'exploitation. Il s'agit essentiellement d'un ensemble de composants
matériels et logiciels doté d'un système d'exploitation intégré lui
permettant de fonctionner de manière indépendante. Il consiste souvent
en une simple combinaison réunissant une carte d'interface réseau (NIC),
un contrôleur de stockage, un certain nombre de baies de disques et un
bloc d'alimentation. Les appareils NAS contiennent deux à cinq disques
durs qui assurent une redondance et un accès rapide aux fichiers. Si le
NAS est souvent considéré comme un mini-serveur, son contrôleur ne gère
que les disques destinés au stockage et ne fonctionne pas comme un
serveur.

En bref, un appareil NAS est une appliance qui se connecte directement
au réseau soit par un une liaison Ethernet câblée (RJ45), soit par
Wi-Fi, créant ainsi un LAN au lieu d'un WAN. Une adresse IP lui est
attribuée, et le transfert de données entre les utilisateurs, les
serveurs et un NAS se fait via TCP/IP. Le NAS fonctionne avec un système
de fichiers traditionnel, soit un système de fichiers de nouvelle
technologie (NTFS) ou NFS pour les services de fichiers à distance et le
partage des données. Tout l'espace de stockage disponible sur l'appareil
est accessible au niveau fichier via un partage de fichiers.

Les appareils NAS fournissent un stockage partagé sous forme de volumes
montés en réseau et utilisent des protocoles tels que NFS et SMB/CIFS.
Lorsqu'il est utilisé pour le stockage partagé, l'appareil NAS relie
plusieurs serveurs à un dispositif de stockage commun. Ces « clusters »
sont souvent utilisés pour le basculement grâce à un volume partagé en
cluster permettant à tous les nœuds du cluster d'accéder aux mêmes
données.

\section{Conclusion}\label{conclusion}

L'utilisation d'une solution de Cloud personnel est un ejeux majeur pour
la souveraineté de ses données personnelles ainsi que du facteur
économique. Cependant cela requière un investissement en temps et en
argent afin de trouver, mettre en place et maintenir cette solution de
stockage dans le temps.

\subsection{Liens}\label{liens}

Ebauche de la recherche bibliographique : {[}Auteur ?{]} \emph{Un cloud
public, qu'est-ce que c'est ?} ({[}sans date{]} a) Inc ({[}sans date{]})
{[}Auteur ?{]} \emph{Les différents types de cloud} ({[}sans date{]})
{[}Auteur ?{]} \emph{Utiliser un {NAS} comme cloud personnel {\textbar}
{Coolblue}} ({[}sans date{]}) {[}Auteur ?{]} \emph{Cloud personnel :
fonctionnement, avantages et mise en place} (2022) Nowtech (2024)
{[}Auteur ?{]} \emph{Transformer son vieux {PC} en serveur {NAS} -
{YouTube}} ({[}sans date{]}) Damien Bernal (2024) Le BENGE (2025)
{[}Auteur ?{]} \emph{{TrueNAS} {Community} {Edition} {\textbar} {Free}
{Open} {Source} {Storage}} ({[}sans date{]}) {[}Auteur ?{]} \emph{Guide
pratique pour l'installation et la configuration de {TrueNAS} {Core}}
(2025) NAS (2023) {[}Auteur ?{]} \emph{Un cloud public, qu'est-ce que
c'est ?} ({[}sans date{]} b) Frandroid (2020) Thiefstep (2025) NAS ★
\emph{et al.} (2023) {[}Auteur ?{]} \emph{Que faire avec un {NAS} pour
la maison ? {\textbar} {Coolblue}} ({[}sans date{]}) {[}Auteur ?{]}
\emph{{NAS} - {Mes} 5 services {Docker} préférés - {Cachem}} (2022)

\subsection{Bibliographie}\label{bibliographie}

Liste des publications :

\phantomsection\label{refs}
\begin{CSLReferences}{0}{1}
\bibitem[\citeproctext]{ref-noauthor_cloud_2022}
{[}Auteur ?{]} \emph{Cloud personnel : fonctionnement, avantages et mise
en place}, 2022. {[}en~ligne{]}. juin 2022.
{[}Consulté~le~6~février~2026{]}. Disponible à
l\textquotesingle adresse~:
\url{https://blog.macway.com/nas/cloud-personnel/}

\bibitem[\citeproctext]{ref-noauthor_guide_2025}
{[}Auteur ?{]} \emph{Guide pratique pour l'installation et la
configuration de {TrueNAS} {Core}}, 2025. {[}en~ligne{]}. octobre 2025.
{[}Consulté~le~6~février~2026{]}. Disponible à
l\textquotesingle adresse~:
\url{https://hetmanrecovery.com/fr/blog/how-to-install-and-configure-truenas-core.htm}

\bibitem[\citeproctext]{ref-noauthor_differents_nodate}
{[}Auteur ?{]} \emph{Les différents types de cloud}, {[}sans date{]}.
{[}en~ligne{]}. {[}Consulté~le~6~février~2026{]}. Disponible à
l\textquotesingle adresse~:
\url{https://www.redhat.com/fr/topics/cloud-computing/public-cloud-vs-private-cloud-and-hybrid-cloud}

\bibitem[\citeproctext]{ref-noauthor_nas_2022}
{[}Auteur ?{]} \emph{{NAS} - {Mes} 5 services {Docker} préférés -
{Cachem}}, 2022. {[}en~ligne{]}. août 2022.
{[}Consulté~le~6~février~2026{]}. Disponible à
l\textquotesingle adresse~:
\url{https://www.cachem.fr/nas-docker-preferes/}

\bibitem[\citeproctext]{ref-noauthor_quest-ce_nodate}
{[}Auteur ?{]} \emph{Qu'est-ce que le cloud ? {\textbar} {Définition} du
cloud}, {[}sans date{]}. {[}en~ligne{]}.
{[}Consulté~le~6~février~2026{]}. Disponible à
l\textquotesingle adresse~:
\url{https://www.cloudflare.com/fr-fr/learning/cloud/what-is-the-cloud/}

\bibitem[\citeproctext]{ref-noauthor_que_nodate}
{[}Auteur ?{]} \emph{Que faire avec un {NAS} pour la maison ? {\textbar}
{Coolblue}}, {[}sans date{]}. {[}en~ligne{]}.
{[}Consulté~le~6~février~2026{]}. Disponible à
l\textquotesingle adresse~:
\url{https://www.coolblue.be/fr/conseils/qu-est-ce-qu-un-nas-pour-la-maison.html}

\bibitem[\citeproctext]{ref-noauthor_transformer_nodate}
{[}Auteur ?{]} \emph{Transformer son vieux {PC} en serveur {NAS} -
{YouTube}}, {[}sans date{]}. {[}en~ligne{]}.
{[}Consulté~le~6~février~2026{]}. Disponible à
l\textquotesingle adresse~:
\url{https://www.youtube.com/watch?v=OKQ6yYMehVQ}

\bibitem[\citeproctext]{ref-noauthor_truenas_nodate}
{[}Auteur ?{]} \emph{{TrueNAS} {Community} {Edition} {\textbar} {Free}
{Open} {Source} {Storage}}, {[}sans date{]}. {[}en~ligne{]}.
{[}Consulté~le~6~février~2026{]}. Disponible à
l\textquotesingle adresse~:
\url{https://www.truenas.com/truenas-community-edition/}

\bibitem[\citeproctext]{ref-noauthor_cloud_nodate}
{[}Auteur ?{]} \emph{Un cloud public, qu'est-ce que c'est ?}, {[}sans
date{]} a. {[}en~ligne{]}. {[}Consulté~le~6~février~2026{]}. Disponible
à l\textquotesingle adresse~:
\url{https://www.redhat.com/fr/topics/cloud-computing/what-is-public-cloud}

\bibitem[\citeproctext]{ref-noauthor_cloud_nodate-1}
{[}Auteur ?{]} \emph{Un cloud public, qu'est-ce que c'est ?}, {[}sans
date{]} b. {[}en~ligne{]}. {[}Consulté~le~6~février~2026{]}. Disponible
à l\textquotesingle adresse~:
\url{https://www.redhat.com/fr/topics/cloud-computing/what-is-public-cloud}

\bibitem[\citeproctext]{ref-noauthor_utiliser_nodate}
{[}Auteur ?{]} \emph{Utiliser un {NAS} comme cloud personnel {\textbar}
{Coolblue}}, {[}sans date{]}. {[}en~ligne{]}.
{[}Consulté~le~6~février~2026{]}. Disponible à
l\textquotesingle adresse~:
\url{https://www.coolblue.be/fr/conseils/nas-cloud-personnel.html}

\bibitem[\citeproctext]{ref-damien_bernal_disque_2024}
DAMIEN BERNAL, 2024. \emph{Disque {Dur}, {Cloud}, {NAS} ... {Que}
{CHOISIR} pour stocker, {Sauvegarder} et {Protèger} vos {Photos} \&
{Vidéos}} {[}en~ligne{]}. mars 2024. {[}Consulté~le~6~février~2026{]}.
Disponible à l\textquotesingle adresse~:
\url{https://www.youtube.com/watch?v=3dLMRQiXSGg}

\bibitem[\citeproctext]{ref-frandroid_opter_2020}
FRANDROID, La rédaction, 2020. \emph{Opter pour un {NAS} : oui, mais
quel modèle choisir ?} {[}en~ligne{]}. février 2020.
{[}Consulté~le~6~février~2026{]}. Disponible à
l\textquotesingle adresse~:
\url{https://www.frandroid.com/guide-dachat/672920_guide-achat-meilleurs-nas}

\bibitem[\citeproctext]{ref-hpe_quest-ce_nodate}
HPE, {[}sans date{]}. \emph{Qu'est-ce que le stockage connecté au réseau
({NAS}) ? {\textbar} {Glossaire}} {[}en~ligne{]}.
{[}Consulté~le~6~février~2026{]}. Disponible à
l\textquotesingle adresse~:
\url{https://www.hpe.com/fr/fr/what-is/nas.html}

\bibitem[\citeproctext]{ref-inc_solutions_nodate}
INC, QNAP Systems, {[}sans date{]}. \emph{Solutions {QNAP} pour la
maison {\textbar} {Utilisez} un {NAS} en tant que cloud privé personnel
et centre multimédia} {[}en~ligne{]}. {[}Consulté~le~6~février~2026{]}.
Disponible à l\textquotesingle adresse~:
\url{https://www.qnap.com/solution/homeusers/fr-fr}

\bibitem[\citeproctext]{ref-le_benge_avez-vous_2025}
LE BENGE, 2025. \emph{Avez-vous vraiment besoins d'un {NAS} ?}
{[}en~ligne{]}. juillet 2025. {[}Consulté~le~6~février~2026{]}.
Disponible à l\textquotesingle adresse~:
\url{https://www.youtube.com/watch?v=CvG3l3JSISg}

\bibitem[\citeproctext]{ref-nas__comparatif_2023}
NAS ★, ★ Forum des, CONTACT, PROPOS, A., NAS \& STOCKAGENAS \&
STOCKAGERETROUVEZ NOS ARTICLES, tests, NAS 2025, Comparatif, 7.2, Dsm,
250 €, NAS à moins de, NAS ?, Comment choisir un, D'ACHAT, NAS-Guide,
NAS, 10 conseils sécurité pour votre, NAS, Tests de, NAS, Tutos,
SYNOLOGY, QNAP, ASUSTOR, NAS, Tous les articles sur les,
HIGH-TECHHIGH-TECHCETTE CATÉGORIE EST TRÈS GÉNÉRALISTE ET NOUS VOUS
INVITONS À DÉCOUVRIR LES SOUS-CATÉGORIES : DOMOTIQUE, Réseau, DOMOTIQUE,
RÉSEAUX, VIDÉO, Jeux, TABLETTE, Mobile \&, LOGICIELSLOGICIELSIL N'Y A
PAS QUE LES NAS DANS LA VIE\ldots{} ICI, nous vous proposons de
découvrir (ou redécouvrir) des applications et logiciels qui nous ont
vraiment plu Zorin OS : Donner une seconde vie à son PC/MacBloquer les
publicités YouTube grâce à votre NAS et iSponsorBlockTVExporter et
sauvegarder ses messages iPhone (iMessage) en 5 minutes,
LOISIRSLOISIRSON NE TESTE PAS QUE LES NAS SUR CACHEM. NOUS SOMMES AUSSI
PASSIONNÉS PAR D'AUTRES TECHNOLOGIES, c'est la raison de cette section
Loisirs : Drone, TUTOSTUTOSRETROUVEZ ICI TOUS NOS TUTORIELS (AUTRES QUE
LES NAS) : SAUVEGARDE, sécurité, FOR, Search, LES BAIES : LE NOMBRE
D'EMPLACEMENTS POUR LES DISQUES DURS OU SSD EST TRÈS IMPORTANT. PLUS
VOUS DISPOSEZ DE BAIES, plus vous avez de possibilités de configuration
RAID et plus de capacité de stockage, BESOINS, Le RAID : Le système de
répartition des données sur plusieurs disques offre une tolérance aux
pannes et une continuité de service Il est important de choisir un NAS
compatible avec les configurations RAID qui répondent à vos, LE
PROCESSEUR : BIEN QUE LA PLUPART DES PROCESSEURS ACTUELS RÉPONDENT AUX
BESOINS DE STOCKAGE, de partage et de sécurité des fichiers, LA MÉMOIRE
VIVE : LA QUANTITÉ DE RAM DISPONIBLE DANS LE NAS DÉTERMINE SA CAPACITÉ À
EXÉCUTER SIMULTANÉMENT PLUSIEURS APPLICATIONS. PLUS LA MÉMOIRE VIVE EST
IMPORTANTE, plus le NAS sera performant dans les tâches multitâches, LE
RÉSEAU : TOUS LES NAS ACTUELS SONT ÉQUIPÉS À MINIMA D'UN PORT RÉSEAU
GIGABIT (1 GB/S). CERTAINS MODÈLES OFFRENT MÊME DEUX PORTS OU PLUS,
permettant l'amélioration des performances Les NAS plus récents
également du Multi-Gig, NAS, Le prix : Le budget est un facteur crucial
lors du choix d'un NAS Il s'agit d'un investissement à long terme qui
accompagnera vos besoins de stockage pendant de nombreuses années Il est
donc important de trouver un équilibre entre les fonctionnalités
offertes et le coût du, INDISPONIBLE, Plex{]} Plex est actuellement,
5.1.2.RE51, Adm, 5.1.2.RE31, Attention à la mise à jour, DISQUE SYSTÈME
DU NAS, SSD de 256 Go ou 512 go ?, PRÉFÉRENCE, cherche{]} Disque dur WD
red de, NOUVEAU, Et un p'ti, TOS6, Qbittorrent bloqué SSRF, 13 (2013)
8GO?, Quelle distribution linux pour un vieux macbook air, TOUS, Bonjour
a, BETA 2, QuTS hero h6 0 0 3382 build 20260122 Public, ÉVOLUTIF,
Paradise dans Test Beelink ME Pro : un NAS hybride performant et, NAS
(SYNOLOGY), zer dans Installer LanguageTool sur un, ÉVOLUTIF, Lestat
dans Test Beelink ME Pro : un NAS hybride performant et, ÉVOLUTIF, HeC
dans Test Beelink ME Pro : un NAS hybride performant et, ÉVOLUTIF, Yann
dans Test Beelink ME Pro : un NAS hybride performant et, CHROME,
Z.-Image dans Capture d'écran facile avec, BLOGMOTION, CONNECT, Emmaüs,
PUBLICS, Films Tous, CENTER, MyQNAP App, UNSIMPLECLIC, ACCUEIL,
STOCKAGE, NAS \&, HIGH-TECH, LOGICIELS, TUTOS, NAS ☆, ☆ Forum des, WEB
2.0, Catégories Blogosphère Bons Plans Cybersécurité Jeux Vidéo Loisirs
Mobile \& Tablette Partage Réseaux Sociaux SEO Société Tutos,
BLOGOSPHÈRE, PLANS, Bons, CYBERSÉCURITÉ, VIDÉO, Jeux, LOISIRS, TABLETTE,
Mobile \&, PARTAGE, SOCIAUX, Réseaux, SEO, SOCIÉTÉ, TUTOS et 2.0, Web,
2023. \emph{Comparatif meilleur {NAS} 2026 - {Cachem}} {[}en~ligne{]}.
juin 2023. {[}Consulté~le~6~février~2026{]}. Disponible à
l\textquotesingle adresse~:
\url{https://www.cachem.fr/meilleurs-nas-comparatif/}

\bibitem[\citeproctext]{ref-nas_quel_2023}
NAS, MS, 2023. \emph{Quel {OS} choisir pour un {NAS} {DIY} ?}
{[}en~ligne{]}. 2023. {[}Consulté~le~6~février~2026{]}. Disponible à
l\textquotesingle adresse~:
\url{https://monserveurnas.com/quel-os-nas-diy-choisir/}

\bibitem[\citeproctext]{ref-nowtech_5_2024}
NOWTECH, 2024. \emph{5 raisons de prendre un {NAS} en 2024}
{[}en~ligne{]}. février 2024. {[}Consulté~le~6~février~2026{]}.
Disponible à l\textquotesingle adresse~:
\url{https://www.youtube.com/watch?v=cBNfZREHGNE}

\bibitem[\citeproctext]{ref-thiefstep_understanding_2025}
THIEFSTEP, 2025. \emph{Understanding how to pick an {OS} for a
{NAS}/{Server}} {[}en~ligne{]}. Reddit \{Post\}. avril 2025.
{[}Consulté~le~6~février~2026{]}. Disponible à
l\textquotesingle adresse~:
\url{https://www.reddit.com/r/homelab/comments/1k5vc07/understanding_how_to_pick_an_os_for_a_nasserver/}

\end{CSLReferences}

\end{document}
